\documentclass{article}

\usepackage[a4paper]{geometry}
\usepackage[spanish]{babel}
\usepackage{xcolor}
\usepackage{placeins}

\usepackage{mathbbol}
\usepackage{amsmath}
\usepackage{amsfonts}
\usepackage{hyperref}
\usepackage{graphicx}
\usepackage{subcaption}

\usepackage{algorithm}
\usepackage{algpseudocode}

% Cambiar 'Cuadro' -> 'Tabla'
\addto\captionsspanish{
    \renewcommand{\tablename}{Tabla}
}

\begin{document}

\begin{center}
    {\Large Aprendizaje Automático para Datos en Grafos} \\
    {\LARGE \textbf{Laboratorio 5}} \\
    \vspace{2em}
    \begin{minipage}{0.45\textwidth}
        \centering
        Graciana Castro \\
        4.808.848-2 \\
        gcastro@fing.edu.uy
    \end{minipage}
    \hfill
    \begin{minipage}{0.45\textwidth}
        \centering
        Julian O'Flaherty \\
        6.285.986-9 \\
        julian.o.flaherty@fing.edu.uy
    \end{minipage}
\end{center}


\section{Introducción}

\section{Graph Convolution}
Se define la convolución en el grafo como $$\mathbf{Y} = \sum_{k=0}^{K} \mathbf{S}^k \mathbf{X} \mathbf{H}_k,$$ donde $\mathbf{Y} \in \mathbb{R}^{N \times F_{\text{out}}}$, $\mathbf{S} \in \mathbb{R}^{N \times N}$ es la matriz de soporte, $\mathbf{X} \in \mathbb{R}^{N \times F_{\text{in}}}$ es la matriz de características de entrada, y $\mathbf{H}_k \in \mathbb{R}^{F_{\text{in}} \times F_{\text{out}}}$ es la matriz de coeficientes del filtro. $\mathbf{S}_{i,j}^k$ será distinta de cero si hay un camino de largo $k$ entre los nodos $i$ y $j$. Por lo tanto, la convolución en el grafo considera la información de los nodos que están a una distancia máxima de $K$ saltos.

En particular, trabajamos con el caso donde la matriz de soporte es la matriz de adyacencia $\mathbf{A}$ del grafo. En este caso, la convolución en el grafo se puede interpretar como una combinación lineal de las características de los nodos vecinos hasta una distancia máxima de $K$ saltos. La convolución queda entonces: 

$$\mathbf{Y} = \sum_{k=0}^{K} \mathbf{A}^k \mathbf{X} \mathbf{H}_k.$$

Si tomamos $\mathbf{Z}_k = \mathbf{A}^k \mathbf{X}$, podemos definir cada $\mathbf{Z}_k$ a partir de $\mathbf{Z}_{k-1}$ como:
$$\mathbf{Z}_k = \mathbf{A} \mathbf{Z}_{k-1},$$
con $\mathbf{Z}_0 = \mathbf{X}$. De esta forma, podemos implementar la convolución en el grafo de manera eficiente. 

\subsection{Implementaciones}
Se probaron dos implementaciones de la convolución en el grafo. Por un lado, realizamos una implementación manual de la clase que denominamos \texttt{GCNLayer}. Esta clase recibe como parámetros la dimensión de entrada $F_{\text{in}}$, la dimensión de salida $F_{\text{out}}$ y el tamaño del filtro $K$. En el método \texttt{forward}, se calcula la convolución en el grafo recorriendo los filtros y propagando la información a través del grafo utilizando el método \texttt{propagate} de PyTorch Geometric.

También se probó la implementación de la clase \texttt{TAGConv} de PyTorch Geometric, que realiza la misma operación de convolución en el grafo. Esta clase también recibe como parámetros la dimensión de entrada $F_{\text{in}}$, la dimensión de salida $F_{\text{out}}$ y el tamaño del filtro $K$. 

Para verificar que ambas implementaciones son correctas, se compararon los resultados obtenidos con ambas clases tomando una señal de entrada aleatoria y una matriz de adyacencia de un grafo pequeño, y se compararon con el resultado de la operación de convolución en el grafo calculada manualmente como $\mathbf{Y} = \sum_{k=0}^{K} \mathbf{A}^k \mathbf{X}$. 

Se tiene que $$\mathbf{A} = \begin{bmatrix}
    0 & 1 & 0 & 0\\
    1 & 0 & 1 & 1\\
    0 & 1 & 0 & 1\\
    0 & 1 & 1 & 0
\end{bmatrix} , \ \ \ \ \mathbf{X} = \begin{bmatrix}
    0 & 1 \\
    2 & 3 \\
    4 & 5 \\
    6 & 7
\end{bmatrix}$$

por lo que el resultado de la operación manual es

$$\mathbf{Y}_{\text{ref}} = \begin{bmatrix}
    28 & 38 \\
    72 & 94 \\
    62 & 80 \\
    62 & 80
\end{bmatrix}.$$

En la figura \ref{fig:resultados_gcn_tag} se muestran los resultados obtenidos con ambas implementaciones, que coinciden con el resultado de la operación manual.

\begin{figure}[htb!]
    \centering
    \includegraphics[width=0.5\textwidth]{imagenes/Resultados_GCN_TAG.png}
    \caption{Resultados obtenidos con ambas implementaciones de la convolución en el grafo.}
    \label{fig:resultados_gcn_tag}
\end{figure}

\section{Graph Perceptron}


\FloatBarrier
\bibliography{refs.bib}
\bibliographystyle{plain}

\end{document}