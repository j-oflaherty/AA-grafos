\documentclass{article}

\usepackage[a4paper]{geometry}
\usepackage[spanish]{babel}

\usepackage{hyperref}
\usepackage{graphicx}



\begin{document}

\begin{center}
    {\Large Aprendizaje Automático para Datos en Grafos} \\
    {\large \textbf{Propuesta de Proyecto Final}} \\
    {\LARGE \textbf{Redes Neuronales en Grafos para visión por computadora}} \\
    \vspace{2em}
    \begin{minipage}{0.45\textwidth}
        \centering
        Graciana Castro \\
        4.808.848-2 \\
        gcastro@fing.edu.uy
    \end{minipage}
    \hfill
    \begin{minipage}{0.45\textwidth}
        \centering
        Julian O'Flaherty \\
        6.285.986-9 \\
        julian.o.flaherty@fing.edu.uy
    \end{minipage}
\end{center}

\section{Introducción}

Las imágenes son un tipo de datos con estructura espacial, la cual podemos representar con un grafo. Por ejemplo, existen métodos clásicos de segmentación
de imágenes que se basan construir un grafo y aplicarle algoritmos para obtener el corte mínimo del mismo~\cite{wiki:graphcuts}. 
Dados los avances en el aprendizaje profundo y los desarrollos en las redes neuronales en grafos, es natural postular la utilidad de estas para tareas
de visión por computadora. En este trabajo se realizará un pequeño relevamiento del estado del arte del uso de GNNs para tareas de visión por computadora,
enfocandose en tareas de clasificación y segmentación de imagenes.

El objetivo de este proyecto será responder la pregunta \textbf{¿Tienen lugar las GNNs en la tearea de Visión por computadora? }

\section{Propuesta de trabajo}

\subsection{Relevamiento del estado del arte}

La primera etapa del proyecto consistirá en un relevamiento del estado del arte, analizando trabajos que hayan explorado esta idea. Por lo que 
hallamos preliminarmente, el desafío se divide en dos etápas:
\begin{itemize}
    \item Diseño y construcción del grafo
    \item Arquitectura y entrenamiento de la GNN
\end{itemize}
En el relevamiento se apuntara a cubrir varias estratégias de construcción del grafo y arquitecturas, buscando analizar que combinación obtiene los mejores
resultados.

Entre los trabajos que encontramos, destacan los siguientes:
\begin{itemize}
    \item \textbf{VisionGNN~\cite{han2022visiongnnimageworth}}: este paper propone una estrategia de creación de grafo tomando patches de a la imagen como 
    nodos y conectandolos en base a su contenido.
    \item \textbf{Superpixel Image Classification with Graph Attention Networks~\cite{avelar2020superpixelimageclassificationgraph}}: este paper propone una estrategia de creación de grafo tomando superpixels de la imagen como nodos y defininiendo una estrategia de conexión basada en el contenido de los mismos.
\end{itemize}

\subsection{Implementación y datasets}

Una vez seleccionados los modelos y estratégias a usar, se recopilaran las implementaciones en un repositorio que permita la rápida evaluación de los 
métodos. Se desarrollara en Python, utilizando la librería torch-geometric para los modelos de aprendizaje profundo. El repositorio se implementará
tal que los bloques de creación de grafos y los modelos sean intercambiables y combinables, permitiendo la rápida iteración y comparación.

Para trabajar con un nivel de computo manejable, se trabajara sobre los siguientes datasets:
\begin{itemize}
    \item MNIST~\cite{deng2012mnist}: dataset de dígitos escritos a mano. Actuara como dataset de debugging, contando con imágenes pequeñas y simples de clasificar,
    permitiendo la rápida iteración y validación de modelos.
    \item STL-10~\cite{pmlr-v15-coates11a}: dataset de imágenes de objetos de 10 clases (500 imagenes de entrenamiento por clase).
    \item ImageNet-100~\cite{ILSVRC15}: subset de ImageNet con 100 clases (en vez de las 10k).
    \item Pascal VOC0712~\cite{everingham2010pascal}: dataset de clasificación, detection y segmentación de objetos. Cuenta con 20 clases.
\end{itemize}

\section{Entregables}

El entregable del proyecto será un informe con los resultados obtenidos, haciendo una comparativa de los resultados obtenidos con cada estrategia. Se sumará
a la comparación, resultados obtenidos para CNNs del mismo tamaño\footnote{Se excluyen los VITs, dado que su arquitectura destaca en grandes volúmenes de datos}.
También se hara una comparativa con la construcción del grafo básica, conectando simplemente patches vecinos, buscando medir el impacto de la construcción
del grafo en el modelo final. Se hara también un analisis computacional, tratando de medir la escalabilidad de estos métodos.

También se entregará el código de investigación utilizado, donde se implementen los modelos y estrategias de construcción de grafo.

\bibliographystyle{plain}
\bibliography{refs}

\end{document}
